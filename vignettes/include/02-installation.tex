\section{Installation}
\label{sec:installation}

In this section, we will describe the various ways that one can build \thispackage.



\subsection{Installing pbdPROF without a System Installation of PAPI}

This is the default method of installation.  Here, the \PAPI package will automatically be built first as a static library, and then the \thispackage package will be built and linked against that static library.  This is the best option if you do not already have a system installation of \PAPI available, or you are making changes to \thispackage (and thus are rebuilding frequently).

\begin{Command}
R CMD INSTALL pbdPAPI_0.1-0.tar.gz
\end{Command}
and using the \pkg{devtools} package:
\begin{lstlisting}
library(devtools)
install_github(username="wrathematics", repo="pbdPAPI")
\end{lstlisting}




\subsection{Linking pbdPROF with an Existing System Installation of PAPI}

To link with an external installation of \PAPI, from the command line, execute:
\begin{Command}
R CMD INSTALL pbdPAPI_0.1-0.tar.gz \ --configure-args="--enable-system-papi \ --with-papi-home=location/of/PAPI/install"
\end{Command}
and using the \pkg{devtools} package:
\begin{lstlisting}
library(devtools)
install_github(username="wrathematics", repo="pbdPAPI", args="--configure-args='--enable-system-papi --with-papi-home=location/of/PAPI/install'")
\end{lstlisting}
